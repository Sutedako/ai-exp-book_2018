%!TEX root = ../main.tex

\setcounter{section}{0}

\parindent = 0pt

\section*{Deep Learning用VM (Linux)構築手順}
1. \url{http://portal.azure.com/}にアクセスして,マイクロソフトアカウントの{\bf[ユーザー名]}と{\bf[パスワード]}を入力して{\bf[サインイン]}をクリックする.

\begin{figure}[ht]
	\begin{center}
		\includegraphics[height=6cm]
		{images/YamasakiLab/azure/pic01.eps}
	\end{center}
\end{figure}

2. ポータル画面が表示されたら,{\bf[+新規]}を選択する.

\begin{figure}[ht]
	\begin{center}
		\includegraphics[height=6cm]
		{images/YamasakiLab/azure/pic02.eps}
	\end{center}
\end{figure}

\newpage

3.Deep Learning Virtual Machineを検索し,{\bf[作成]}をクリックする.

\begin{figure}[ht]
	\begin{center}
		\includegraphics[height=5cm]
		{images/YamasakiLab/azure/pic03.eps}
	\end{center}
\end{figure}

\begin{figure}[ht]
	\begin{center}
		\includegraphics[height=5cm]
		{images/YamasakiLab/azure/pic04.eps}
	\end{center}
\end{figure}

4. VMの基本設定を行う.

\parindent = 2pt

\textcircled{\scriptsize 1} {\bf[Name]}にVM名を入力し,OSをWindowsからLinuxに変更する.

\textcircled{\scriptsize 2} {\bf[User name]},{\bf[Password]},{\bf[Confirm password]}を埋める.
{\bf[サブスクリプション]}は特には変更しなくてよい.

\textcircled{\scriptsize 3} {\bf[リソース グループ]}は新規作成を選択し,好きなリソースグループ名を記入する.

\textcircled{\scriptsize 4} {\bf[場所]}でサーバを作成するリージョンを選択する.
リージョンごとにサーバの値段が異なり,リージョンによってはGPU付きのNシリーズのVMを構築できないので注意.
値段が安いため,米国東部,米国中北部,米国西部2あたりを勧める.

\textcircled{\scriptsize 5} 以上を埋めたら,{\bf[OK]}をクリックして次に進む.

\parindent = 0pt

\begin{figure}[ht]
	\begin{center}
		\includegraphics[height=5cm]
		{images/YamasakiLab/azure/pic05.eps}
	\end{center}
\end{figure}

\newpage

5. {\bf[Virtual machine size]}をクリックしてVMのサイズを選択する.
各リージョンで作成できるvCPU数は最大48 (GPU 8つ分)である.
サイズは後からでも変更できる.
サイズを選択したら,{\bf[OK]}をクリックする.

\begin{figure}[ht]
	\begin{center}
		\includegraphics[height=5cm]
		{images/YamasakiLab/azure/pic06.eps}
	\end{center}
\end{figure}

6. 最終検証が行われる.
失敗した場合は現在選択しているリージョンで作成できる最大vCPU数を超えている可能性があるため,4に戻って違うリージョンを選択し直す.
検証に成功したら,{\bf[OK]}をクリックする.

\begin{figure}[ht]
	\begin{center}
		\includegraphics[height=5cm]
		{images/YamasakiLab/azure/pic07.eps}
	\end{center}
\end{figure}

7. {\bf[作成]}をクリックする.VMのデプロイが開始され,完了するまで数十分かかる.

\begin{figure}[ht]
	\begin{center}
		\includegraphics[height=5cm]
		{images/YamasakiLab/azure/pic08.eps}
	\end{center}
\end{figure}

\newpage

8. デプロイが完了したら,ダッシュボードに作成されたリソースタイルの{\bf[表示数を増やす]}をクリックし,ディスプレイマークの{\bf[仮想マシン]}をクリックする.

\begin{figure}[ht]
	\begin{center}
		\includegraphics[height=5cm]
		{images/YamasakiLab/azure/pic09.eps}
	\end{center}
\end{figure}

\begin{figure}[ht]
	\begin{center}
		\includegraphics[height=5cm]
		{images/YamasakiLab/azure/pic10.eps}
	\end{center}
\end{figure}


すると,以下のようなVMの基本情報が載った画面(以降,VM基本画面と呼ぶ)が表示される.
この画面でVMの起動と停止等を行う(最初は起動してある).
この画面はよく使用するので,ダッシュボードにピン留めしておくとよい.

\begin{figure}[ht]
	\begin{center}
		\includegraphics[height=5cm]
		{images/YamasakiLab/azure/pic11.eps}
		%\caption{VM基本画面}
	\end{center}
\end{figure}

\newpage

9. VM基本画面の左上にある{\bf[接続]}をクリックするとVMに接続するsshコマンドが表示されるため,これを参考にVMに接続する.

\begin{figure}[ht]
	\begin{center}
		\includegraphics[height=5cm]
		{images/YamasakiLab/azure/pic12.eps}
	\end{center}
\end{figure}

\parindent = 10pt

\section*{IPアドレスの固定化}
デフォルトではIPアドレスは動的であり,VMを停止するたびにIPが変更されてしまう.
それを防ぐにはIPアドレスを静的にする.
VMを起動したままの状態でVM基本画面の{\bf[パブリックIPアドレス]}の下のIPアドレスをクリックする.

\begin{figure}[ht]
	\begin{center}
		\includegraphics[height=5cm]
		{images/YamasakiLab/azure/pic13.eps}
	\end{center}
\end{figure}

{\bf[割り当て]}を動的から静的に変更する.{\bf[保存]}をクリックして変更を完了する.

\begin{figure}[ht]
	\begin{center}
		\includegraphics[height=5cm]
		{images/YamasakiLab/azure/pic14.eps}
	\end{center}
\end{figure}

\newpage

\section*{ストレージ変更}
デフォルトではストレージは50GB程度しかないので,ストレージを増やすことを勧める.
まずVM基本画面で{\bf[停止]}をクリックしてVMを停止する.
VMが停止したら,左側の{\bf[ディスク]}をクリックする.

\begin{figure}[ht]
	\begin{center}
		\includegraphics[height=5cm]
		{images/YamasakiLab/azure/pic15.eps}
	\end{center}
\end{figure}

{\bf[OSディスク]}を選択し,{\bf[サイズ]}を欲しいストレージ分だけ変更する.
一度増やしたストレージは減らすことができないので注意.
ストレージを変更したら{\bf[保存]}をクリックして変更を完了する.

\begin{figure}[ht]
	\begin{center}
		\includegraphics[height=5cm]
		{images/YamasakiLab/azure/pic16.eps}
	\end{center}
\end{figure}

\begin{figure}[ht]
	\begin{center}
		\includegraphics[height=5cm]
		{images/YamasakiLab/azure/pic17.eps}
	\end{center}
\end{figure}

\newpage

\section*{サイズ変更}
VM作成後にVMのvCPU数やGPU数を変更できる.
VM基本画面の左側の{\bf[サイズ]}をクリックし,変更を行う.
先ほどでも述べた通り,一つのリージョンで作成できるvCPU数は最大48 (GPU 8つ分)であり,それを超える場合はサイズを増やすことができないので注意.

\begin{figure}[ht]
	\begin{center}
		\includegraphics[height=5cm]
		{images/YamasakiLab/azure/pic18.eps}
	\end{center}
\end{figure}

\section*{ユーザ追加}
一つのVMで複数人が作業する場合はユーザを追加すると良い.
VM基本画面の左側を下にスクロールし,{\bf[パスワードのリセット]}をクリックして新規ユーザのユーザ名とパスワードを設定する.

\begin{figure}[ht]
	\begin{center}
		\includegraphics[height=5cm]
		{images/YamasakiLab/azure/pic19.eps}
	\end{center}
\end{figure}